\section*{How to Read this Document}
This document makes extensive use of links, references and notices in the page margins to detail additional information that can be useful while following the guide. \\

\warningnote{A warning notice indicates a potential hazard. If care is not taken to adhere to the safety precautions, damage may be done to snickerdoodle.}

\noindent
Warnings and cautions will be clearly visible in either the body of the text or in the margin and must be paid close attention while following the guided steps. \\

\margininfonote{Warning, caution and informational notices, such as this one, may also be found in the margin.}


\cautionnote{A caution indication denotes a process that requires special attention. If the caution is not exercised and the process not adhered to, failure may result and/or potential damage to snickerdoodle.}

\subsection*{Keywords}
Keywords and important terms are shown in \textit{italicized} type. Additional important information can be found in the margins of text with superscript notation\sidenote{Margin notes, such as this one, reference the body content and highlight technical details or references for further information.}. \\

\noindent
Navigation of menus and directories are shown using \textbf{\textit{bold italicized}} type. Any hierarchical navigation is shown using an arrow to denote a \menupath{Parent, child} relationship. \\

\noindent
\texttt{Teletype} text is used to highlight inputs, variables and system files within the host environment.\newpage